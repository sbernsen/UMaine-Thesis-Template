%\copyrightpage[...]{...}		% Optional, comment out or delete if undesired

\begin{abstract}

This is the abstract. Sentence two. Formatting is easy when you use \LaTeX and it's easy to control. It excels in the math environement but tables can sometimes require more effort. Fortunately, it's open source (i.e. free), platform independent, and there's a big user community. There's a list of resources at the end.


\end{abstract}

%\begin{layabstract}{...}	% Replace the ... with the list of keywords
% Lay abstract text here, either typed in directly or included using an `\input{}' command
%\end{layabstract}

% The optional preface, dedication, and acknowledgements environments are included similar to the abstract environment

%%%%%%%%%%%%%%%%%%%%%%%%%%%% Preface %%%%%%%%%%%%%%%%%%%%%%%%%%%%%%%%%
%\begin{preface}
% Preface text here
%\end{preface} 

%%%%%%%%%%%%%%%%%%%%%%%%%%% Dedication %%%%%%%%%%%%%%%%%%%%%%%%%%%%%%%
%\begin{dedication}
% Dedication text here
%\end{dedication}

%%%%%%%%%%%%%%%%%%%%%%%% Acknowledgements %%%%%%%%%%%%%%%%%%%%%%%%%%%%
%\begin{acknowledgements}
% Acknowledgments text here
%\end{acknowledgements}


%%%%%%%%%%%%%%%%%%%%%%%%%%%%%%%%%%%%%%%%%%%%%%%%%%%%%%%%%%%%%%%%%%%%%%
% Commands for the required lists
\setcounter{page}{2}
\tableofcontents
\listoftables				% Include only if there are tables in the thesis
\listoffigures				% Include only if there are figures in the thesis


%%%%%%%%%%%%%%%%%%%%%%%%%%%%%%%%%%%%%%%%%%%%%%%%%%%%%%%%%%%%%%%%%%%%%%
%% For a section that needs to be included in the front matter insert into PreChapter.tex file. 
\newpage
\addcontentsline{toc}{chapter}{Front Matter Chapter}

% Abbreviations and such need to be in the front matter. The numbering needs to be roman numerals
%\begin{center}
\textbf{LIST OF ABBREVIATIONS}
\end{center}
\begin{multicols}{2}
\noindent 3D - 3-Dimensional\\
\noindent AAR - Accumulation Area Ratio\\
\noindent ASTER - Advanced Spaceborne Thermal Emission and Reflection Radiometer\\
\noindent BC - British Columbia, Canada\\
\noindent CMVS - Clustering View for Multi-view Stereo\\
\noindent CORS - Continuously Operating Reference Station\\
\noindent DEM - Digital Elevation Model\\
\noindent DSLR - Digital Single Lens Reflex\\
\noindent ELA - Equilibrium Line Altitude\\
\noindent FFGR - Foundation For Glacier Research (predecessor of FGER)\\
\noindent FGER - Foundation for Glacier and Environmental Research (funding partner of JIRP)\\
\noindent GCP - Ground Control Point\\
\noindent GIS - Geographical Information System\\
\noindent GPS - Global Positioning System\\
\noindent IMU - Inertial Measurement Unit\\
\noindent JIRP - Juneau Icefield Research Program\\
\noindent LIA - Little Ice Age\\
\noindent LIDAR - Light Detection And Ranging\\
\noindent m.a.s.l. - Meters Above Mean Sea Level\\
\noindent MP - Mega-Pixel (one million pixels)\\
\noindent PDO - Pacific Decadal Oscillation\\
\noindent PMVS - Patch-based for Multi-view Stereo\\
\noindent RBG - Red, Blue, Green\\
\noindent RMS - Root Mean Square\\
\noindent RTK - Real Time Kinetic\\
\noindent SfM - Structure from Motion\\
\noindent SIFT - Scale Invariant Feature Transform\\
\noindent SLR - Sea Level Rise\\
\noindent TLS - Terrestrial Laser Scanning\\
\noindent UAV - Unmanned Aerial Vehicle\\
\noindent WAAS - Wide Area Augmentation System\\
\noindent w.e. - Water Equivalent\\
\end{multicols}
\endinput

% If you have other lists which need to be included they go here, possibly using the listof environment
%\begin{listof}{...}		% Replace the ... with name of the things being listed here
% Contents of list
%\end{listof}

% Sets the document spacing and pagestyle.  It is recommended that the `bottom' option be used. 
\mainmatter{bottom} 	

\endinput